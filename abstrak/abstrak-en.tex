\begin{center}
  \large\textbf{ABSTRACT}
\end{center}

\addcontentsline{toc}{chapter}{ABSTRACT}

\vspace{2ex}

\begingroup
% Menghilangkan padding
\setlength{\tabcolsep}{0pt}

\noindent
\begin{tabularx}{\textwidth}{l >{\centering}m{3em} X}
  % Ubah kalimat berikut dengan nama mahasiswa
  \emph{Name}     & : & Aufa Nabil Amiri                                             \\

  % Ubah kalimat berikut dengan judul tugas akhir dalam Bahasa Inggris
  \emph{Title}    & : & \emph{ Automatic Indonesian Hoax News Detection Using BERT } \\

  % Ubah kalimat-kalimat berikut dengan nama-nama dosen pembimbing
  \emph{Advisors} & : & 1. Reza Fuad Rachmadi, S.T., M.T., Ph.D                      \\
                  &   & 2. Prof. Dr. Ir. Mauridhi Hery Purnomo, M. Eng               \\
\end{tabularx}
\endgroup

% Ubah paragraf berikut dengan abstrak dari tugas akhir dalam Bahasa Inggris
\textit{
  Fake news or usually called hoax, is one of things that ofen plaguing Indonesia. With a social media, a fake news can spread wider and faster than ever before. On another note, Indonesian people have quite a high tendencies to share fake news. Based on that note, we are in dire need of a method to detect fake news. This research is using BERT method to automatically detect whether a news can be considered as hoax or not. From a raw text, we are doing a tokenization process before we feed the text to the BERT method. Next, the pooled output of the BERT is being used as the input for Linear Regression, a tested-and-true method for classifying task. Now that we have pass-through all  those steps, we can determine whether a text is a hoax or not. The purpose of this research is to create a machine learning model to help the people to determine whether a text can be considered as hoax or not. And the result is a model to classifying a hoax text with the accuracy of 89\%.
}

% Ubah kata-kata berikut dengan kata kunci dari tugas akhir dalam Bahasa Inggris
\emph{Keywords}: \emph{BERT}, \emph{Hoax}, \emph{Fake News} \emph{Classification}, \emph{Linear Regression}.

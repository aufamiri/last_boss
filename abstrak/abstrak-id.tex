\begin{center}
  \large\textbf{ABSTRAK}
\end{center}

\addcontentsline{toc}{chapter}{ABSTRAK}

\vspace{2ex}

\begingroup
% Menghilangkan padding
\setlength{\tabcolsep}{0pt}

\noindent
\begin{tabularx}{\textwidth}{l >{\centering}m{2em} X}
  % Ubah kalimat berikut dengan nama mahasiswa
  Nama Mahasiswa    & : & Aufa Nabil Amiri                                                   \\

  % Ubah kalimat berikut dengan judul tugas akhir
  Judul Tugas Akhir & : & Deteksi Berita Palsu Otomatis Berbahasa Indonesia Menggunakan BERT \\

  % Ubah kalimat-kalimat berikut dengan nama-nama dosen pembimbing
  Pembimbing        & : & 1. Reza Fuad Rachmadi ST., MT., Ph.D                               \\
                    &   & 2. Prof. Dr. Ir. Mauridhi Hery Purnomo, M. Eng                     \\
\end{tabularx}
\endgroup

% Ubah paragraf berikut dengan abstrak dari tugas akhir
Berita palsu atau yang biasa disebut hoaks adalah suatu yang hal yang sering melanda Indonesia. Dengan adanya sosial media, suatu berita palsu dapat memiliki tingkat penyebaran yang sangat luas. Selain itu, masyarakat Indonesia memiliki tingkat kecenderungan untuk menyebarkan berita palsu yang cukup tinggi. Sehingga, suatu metode pendeteksi berita palsu harus ada. Penelitian ini memanfaatkan algoritma BERT yang digunakan untuk mendeteksi apakah suatu berita adalah berita hoaks atau tidak secara otomatis. Dari suatu teks yang mentah, akan dilakukan tokenisasi sebelum akhirnya dimasukkan ke dalam algoritma BERT. Selanjutnya, keluaran dari BERT akan dijadikan sebagai inputan dari algoritma klasifikasi Linear Regression.  Barulah pada saat ini, kita bisa mendapatkan klasifikasi apakah suatu teks itu berupa berita hoaks atau tidak. Tujuan dari penelitian ini adalah untuk membuat sebuah model yang dapat digunakan untuk melakukan klasifikasi suatu teks apakah termasuk ke dalam berita hoaks atau tidak. Hasil dari penelitian ini adalah model yang dapat mendeteksi berita hoaks dengan tingkat akurasi sebesar 89\%.

% Ubah kata-kata berikut dengan kata kunci dari tugas akhir
Kata Kunci:   BERT, Hoaks, Klasifikasi, Linear Regression.

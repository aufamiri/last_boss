\chapter{TINJAUAN PUSTAKA}
\label{chap:tinjauanpustaka}

% Ubah bagian-bagian berikut dengan isi dari tinjauan pustaka
Demi mendukung penelitian ini, dibutuhkan beberapa teori penunjang sebagai bahan acuan dan referensi. Dengan demikian penelitian ini menjadi lebih terarah.

\section{Berita Palsu}
\label{sec:beritapalsu}

Berita palsu atau biasa dikenal dengan berita hoaks adalah sebuah informasi yang sesungguhnya tidak benar, tetapi dibuat seolah - olah benar adanya \cite{berita_bohong}. Di Indonesia sendiri, hoaks menjadi sebuah masalah tersendiri, hal ini karena masih banyak masyarakat yang langsung mempercayai apapun yang mereka temui di internet tanpa melakukan cek fakta terlebih dahulu.

Ada banyak sekali efek dari berita palsu ini, mulai dari hilangnya reputasi sampai nyawa yang terancam. Salah satu contoh kasus yang cukup parah adalah kerusuhan yang terjadi di Papua, dimana kerusuhan tersebut disebabkan karena adanya hoaks soal ucapan rasialis dari seorang guru SMP kepada muridnya \cite{efek_hoax}.

\section{\textit{Machine Learning}}

\textit{Machine Learning} atau Pembelajaran Mesin adalah salah satu cabang dalam kecerdasan buatan dan ilmu komputer yang menggunakan data dan algoritma untuk meniru manusia dalam mempelajari sesuatu. Untuk saat ini, pembelajaran mesin adalah salah satu ilmu yang cukup sering digunakan oleh berbagai perusahaan
\begin{center}
  \Large \textbf{KATA PENGANTAR}
\end{center}

\addcontentsline{toc}{chapter}{KATA PENGANTAR}

% Ubah paragraf-paragraf berikut dengan isi dari kata pengantar
Puji syukur kehadirat Allah Swt. atas segala limpahan berkah, rahmat, dan hidayah-Nya, penulis dapat menyelesaikan penelitian ini dengan judul \textbf{Deteksi Berita Palsu Otomatis Berbahasa Indonesia Menggunakan BERT}.

Penelitan ini disusun dalam rangka pemenuhan bidang riset di Departmen Teknik Komputer, serta digunakan sebagai persyaratan menyelesaikan pendidikan S1. Penelitian ini dapat terselesaikan tidak lepas dari bantuan sebagai pihak. Oleh karena itu, penulis mengucapkan terima kasih kepada:

\begin{enumerate}[nolistsep]
  \item Keluarga, Ibu, Bapak, dan Adik - Adik tercinta yang telah memberikan dorongan spiritual dan material dalam penyelesaian buku penelitian ini.
  \item Bapak Dr. Supeno Mardi Susiki Nugroho, ST., MT. selaku Kepala Departemen Teknik Komputer, Fakultas Teknologi Elektro dan Informatika Cerdas (FTEIC), Institut Teknologi Sepuluh Nopoember.
  \item Bapak Reza Fuad Rachmadi, ST., MT., Ph. D selaku dosen pembimbing I yang selalu memberikan arahan dan saran selama mengerjakan penelitian tugas akhir ini.
  \item Bapak Prof. Dr. Ir. Mauridhy Hery Purnomo. M. Eng selaku dosen pembimbing II yang selalu memberikan arahan dan saran selama mengerjakan penelitian tugas akhir ini.
  \item Bapak-ibu dosen pengajar Departemen Teknik Komputer atas pengajaran, bimbingan, serta perhatian yang diberikan kepada penulis selama ini.
  \item Seluruh teman - teman dari angkatan e57, Teknik Komputer, Laboratorium B201 Telematika Teknik Komputer ITS, dan Laboratorium B401 Komputasi Multimedia.

\end{enumerate}

\begin{flushright}
  \begin{tabular}[b]{c}
    % Ubah kalimat berikut dengan tempat, bulan, dan tahun penulisan
    Surabaya, Juni 2021 \\
    \\
    \\
    % Ubah kalimat berikut dengan nama mahasiswa
    Aufa Nabil Amiri
  \end{tabular}
\end{flushright}
